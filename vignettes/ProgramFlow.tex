\documentclass[10pt,twoside,]{pinp}

%% Some pieces required from the pandoc template
\providecommand{\tightlist}{%
  \setlength{\itemsep}{0pt}\setlength{\parskip}{0pt}}

% Use the lineno option to display guide line numbers if required.
% Note that the use of elements such as single-column equations
% may affect the guide line number alignment.

\usepackage[T1]{fontenc}
\usepackage[utf8]{inputenc}

% pinp change: the geometry package layout settings need to be set here, not in pinp.cls
\geometry{layoutsize={0.95588\paperwidth,0.98864\paperheight},%
  layouthoffset=0.02206\paperwidth, layoutvoffset=0.00568\paperheight}

\definecolor{pinpblue}{HTML}{185FAF}  % imagecolorpicker on blue for new R logo
\definecolor{pnasbluetext}{RGB}{101,0,0} %


\usepackage{booktabs}
\usepackage{colortbl}

\title{Program Flow in readJDX}

\author[a]{Bryan A. Hanson}

  \affil[a]{Dept. of Chemistry \& Biochemistry, DePauw University;
\url{hanson@depauw.edu}}

\setcounter{secnumdepth}{0}

% Please give the surname of the lead author for the running footer
\leadauthor{}

% Keywords are not mandatory, but authors are strongly encouraged to provide them. If provided, please include two to five keywords, separated by the pipe symbol, e.g:
 

\begin{abstract}

\end{abstract}

\dates{This version was compiled on \today} 

% initially we use doi so keep for backwards compatibility
\doifooter{github.com/bryanhanson/readJDX}
% new name is doi_footer

\pinpfootercontents{Program Flow}

\begin{document}

% Optional adjustment to line up main text (after abstract) of first page with line numbers, when using both lineno and twocolumn options.
% You should only change this length when you've finalised the article contents.
\verticaladjustment{-2pt}

\maketitle
\thispagestyle{firststyle}
\ifthenelse{\boolean{shortarticle}}{\ifthenelse{\boolean{singlecolumn}}{\abscontentformatted}{\abscontent}}{}

% If your first paragraph (i.e. with the \dropcap) contains a list environment (quote, quotation, theorem, definition, enumerate, itemize...), the line after the list may have some extra indentation. If this is the case, add \parshape=0 to the end of the list environment.


This vignette is based on \texttt{readJDX} version 0.3.474.

\hypertarget{program-flow}{%
\section{Program Flow}\label{program-flow}}

\texttt{readJDX} is coded in such a way that it should be easy to add
features. Contributions to improve or expand the package, including pull
requests, are always welcome! Table \ref{tab:PF} shows the overall flow
of the function calls. Only a couple of these functions are exported, so
take a look at the source code for documentation. Be sure to check out
the \emph{MiniDIFDUP\_1} and \emph{MiniDIFDUP\_2} vignettes for
additional information about the JCAMP-DX file structure and how
\texttt{readJDX} functions extract the data.

\begin{table}[!h]

\caption{\label{tab:progFlow}Program Flow.\label{tab:PF}}
\centering
\begin{tabular}{ll}
\toprule
function & input\\
\midrule
\rowcolor{gray!6}  readJDX & file name\\
- findDataTables & character vector: all lines from original file\\
\rowcolor{gray!6}  - extractParams & character vector: just the metadata\\
- processDataTable & character vector: a single VL\\
\rowcolor{gray!6}  - - decompressXYY & character vector: a single VL\\
\addlinespace
- - - decompLines & character vector named with line numbers: a single VL\\
\rowcolor{gray!6}  - - - - getJDXcompression & character vector named with line numbers: a single VL\\
- - - - unSQZ & *list* of character vectors from a single VL; the character vectors are named with the ASDF mode, the list is named with line numbers\\
\rowcolor{gray!6}  - - - - insertDUPS & *list* of character vectors from a single VL; the character vectors are named with the ASDF mode, the list is named with line numbers\\
- - - - - repDUPS & a string of length one (named?)\\
\addlinespace
\rowcolor{gray!6}  - - - - deDIF & *list* of character vectors from a single VL; the character vectors are named with the ASDF mode, the list is named with line numbers\\
- - - - - unDIF & character vector from one line of VL, named by ASDF code\\
\rowcolor{gray!6}  - - - - yValueCheck & *list* of character vectors from a single VL; the character vectors are named with the ASDF mode, the list is named with line numbers\\
\bottomrule
\multicolumn{2}{l}{\textit{Note: }}\\
\multicolumn{2}{l}{VL stands for variable list, as defined in the JCAMP-DX standard. For examples see the *MiniDIFDUP\_1* vignette.}\\
\end{tabular}
\end{table}

%\showmatmethods

\pnasbreak 




\end{document}

